%%%%%%%%%%%%%%%%%%%%%%%%%%%%%%%%%%%%%%%%%
% Beamer Presentation
% LaTeX Template
% Version 1.0 (10/11/12)
%
% This template has been downloaded from:
% http://www.LaTeXTemplates.com
%
% License:
% CC BY-NC-SA 3.0 (http://creativecommons.org/licenses/by-nc-sa/3.0/)
%
%%%%%%%%%%%%%%%%%%%%%%%%%%%%%%%%%%%%%%%%%

%----------------------------------------------------------------------------------------
%	PACKAGES AND THEMES
%----------------------------------------------------------------------------------------

% TODO swap comments in order to generate pauses.
% \documentclass[usenames,dvipsnames]{beamer}
\documentclass[handout,usenames,dvipsnames]{beamer}

\mode<presentation> {

% The Beamer class comes with a number of default slide themes
% which change the colors and layouts of slides. Below this is a list
% of all the themes, uncomment each in turn to see what they look like.

%\usetheme{default}
%\usetheme{AnnArbor}
%\usetheme{Antibes}
%\usetheme{Bergen}
%\usetheme{Berkeley}
%\usetheme{Berlin}
%\usetheme{Boadilla}
%\usetheme{CambridgeUS}
%\usetheme{Copenhagen}
%\usetheme{Darmstadt}
%\usetheme{Dresden}
%\usetheme{Frankfurt}
%\usetheme{Goettingen}
%\usetheme{Hannover}
%\usetheme{Ilmenau}
%\usetheme{JuanLesPins}
%\usetheme{Luebeck}
%\usetheme{Madrid}
%\usetheme{Malmoe}
%\usetheme{Marburg}
%\usetheme{Montpellier}
%\usetheme{PaloAlto}
%\usetheme{Pittsburgh}
%\usetheme{Rochester}
%\usetheme{Singapore}
%\usetheme{Szeged}
\usetheme{Warsaw}

% As well as themes, the Beamer class has a number of color themes
% for any slide theme. Uncomment each of these in turn to see how it
% changes the colors of your current slide theme.

%\usecolortheme{albatross}
%\usecolortheme{beaver}
%\usecolortheme{beetle}
%\usecolortheme{crane}
%\usecolortheme{dolphin}
%\usecolortheme{dove}
%\usecolortheme{fly}
%\usecolortheme{lily}
%\usecolortheme{orchid}
%\usecolortheme{rose}
%\usecolortheme{seagull}
%\usecolortheme{seahorse}
%\usecolortheme{whale}
%\usecolortheme{wolverine}

%\setbeamertemplate{footline} % To remove the footer line in all slides uncomment this line
%\setbeamertemplate{footline}[page number] % To replace the footer line in all slides with a simple slide count uncomment this line

%\setbeamertemplate{navigation symbols}{} % To remove the navigation symbols from the bottom of all slides uncomment this line
}

\usepackage{graphicx} 
\usepackage{booktabs} 
\usepackage[utf8]{inputenc}
\usepackage[polish]{babel}
%\usepackage{polski}
\usepackage{float}
\usepackage{subfig}
\usepackage{amsmath}
\usepackage{amsfonts}
\usepackage{amssymb}
\usepackage{amsthm}

\usepackage{tikz}

\setbeamertemplate{theorems}[numbered]

%----------------------------------------------------------------------------------------
%	TITLE PAGE
%----------------------------------------------------------------------------------------

\title{N-fold integer programming in cubic time} % The short title appears at the bottom of every slide, the full title is only on the title pageIntroduction

\author{Piotr Żuber} % Your name
\institute[University of Warsaw] % Your institution as it will appear on the bottom of every slide, may be shorthand to save space
{
University of Warsaw
\\ % Your institution for the title page
\medskip
\textit{} % Your email address
}
\date{May 5, 2022} % Date, can be changed to a custom date

\begin{document}

\newtheorem{thm}{Theorem}
\newtheorem{defi}{Definition}
\newtheorem*{defi*}{Definition}
\newtheorem*{thm*}{Theorem}
\newtheorem{lm}{Lemma}
\newtheorem*{lm*}{Lemma}
\newtheorem{fakt}{Fact}
\newtheorem*{fakt*}{Fact}

%\newtheorem{ w ni o s e k }[ t w i e r d z e n i e ] { Wniosek }
%\newtheorem{ s t w i e r d z e n i e }[ t w i e r d z e n i e ] { S t w i e r d z e n i e }

%\newtheorem ∗{ d e f i n i c j a }{ D e f i n i c j a }
%\newtheorem ∗{ o z n a c z e n i e }{ O z na c z e ni e }
\renewcommand{\a}{\alpha}
\newcommand{\e}{\varepsilon}
\newcommand{\E}{\mathbb{E}}
\renewcommand{\P}{\mathbb{P}}
\renewcommand{\b}{\beta}
\newcommand{\s}{\sigma}
\renewcommand{\S}{\Sigma}
\renewcommand{\t}{\tau}
\renewcommand{\t}{\theta}
\newcommand{\F}{\Phi}
% \renewcommand{\l}{\lambda}
\newcommand{\g}{\gamma}
\newcommand{\R}{\mathbb{R}}
\newcommand{\Z}{\mathbb{Z}}
\newcommand{\dis}{\displaystyle}
\newcommand{\vect}[1]{\textbf{#1}}
\newcommand{\maxnorm}[1]{\lVert #1 \rVert_{\infty}}
\newcommand{\norm}[1]{\lVert #1 \rVert}
\newcommand{\bigO}{\mathcal{O}}
\newcommand{\cord}{\sqsubseteq}
\newcommand{\graver}{\mathcal{G}}
\renewcommand{\figurename}{Figure}

\begin{frame}
\titlepage % Print the title page as the first slide
\end{frame}

% Source: https://dl.acm.org/doi/10.1145/3397484

%----------------------------------------------------------------------------------------
%	PRESENTATION SLIDES
%----------------------------------------------------------------------------------------

%------------------------------------------------ % Sections can be created in order to organize your presentation into discrete blocks, all sections and subsections are automatically printed in the table of contents as an overview of the talk
%------------------------------------------------

% \subsection{Subsection Example} % A subsection can be created just before a set of slides with a common theme to further break down your presentation into chunks

\begin{frame}{Paper}
Raymond Hemmecke, Schmuel Onn, Lyubov Romanchuk -- N-fold integer programming in cubic time 
\textit{Parameterized Approximation Schemes for Independent Set of Rectangles and Geometric Knapsack}
\end{frame}

\begin{frame}{Introduction}
    \begin{defi}[Bimatrix]
        Matrix $A$ is a $(r, s)\times t$ bimatrix if $A = \left(\begin{matrix}A_1 \\ A_2 \end{matrix}\right)$ for $A_1 \in \Z^{r \times t}$ and $A_2 \in \Z^{s \times t}.$
    \end{defi}
\end{frame}

\begin{frame}{Introduction}
    \begin{block}{Bimatrix}
        $A = \left(\begin{matrix}A_1 \\ A_2 \end{matrix}\right) \in (r,s) \times t.$
    \end{block}
    \pause
    \begin{defi}[N-fold product]
        For any natural $n$ and fixed bimatrix $A,$ the $n$-fold product $A^{(n)} \in (r+ns) \times nt$ of $A$ is defined as
        $$
            A^{(n)} = \left(
                \begin{matrix}
                    A_1    & A_1    & \cdots & A_1    \\
                    A_2    & 0      & \cdots & 0      \\
                    0      & A_2    & \cdots & 0      \\
                    \vdots & \vdots & \ddots & \vdots \\
                    0      & 0      & \cdots & A_2
                \end{matrix}
            \right)
        $$
    \end{defi}
\end{frame}

\begin{frame}{Introduction}
    \begin{defi}[N-fold integer programming]
        Given 
        \begin{itemize}
            \item<1-> weight vector $\vect{w} \in \Z^{nt},$
            
            \item<2-> bound vectors $\vect{l}, \vect{u} \in \Z^{nt},$
            
            \item<3-> $b \in \Z^{r+ns},$
            
            \item<4-> $n$-fold product $A^{(n)}$
        \end{itemize}
        $n$-fold integer programming is the following problem
        \begin{equation*}
            \min\{\vect{w}\vect{x} : A^{(n)}\vect{x} = \vect{b}, \vect{l} \leqslant \vect{x} \leqslant \vect{u}, \vect{x} \in \Z^{nt}\}.
        \end{equation*}
    \end{defi}
\end{frame}



\begin{frame}{Introduction}
    \begin{itemize}
        \item<1-> The fastest algorithm for $n$-fold integer programming runs in $\bigO \left(n^{g(A)}L\right)$ time, where $g(A)$ is a Graver complexity of the bimatrix $A$ and $L$ in the length of problem input...
        \item<2-> however, the Graver complexity can be quite large -- there are $\bigO(m)\times \bigO(m)$ matrices with $\Omega(2^m)$ complexity.
        \item<3-> This algorithm can be improved to $\bigO(n^3L)$ regardless of $A.$
    \end{itemize}
\end{frame}

\begin{frame}{Graver complexity}
    \begin{defi}[Conformal order]
        Denote the conformal order relation by $\cord.$ For $\vect{u}, \vect{v} \in \Z^n,$ we say $\vect{u} \cord \vect{v},$ if for every $i:$
        \begin{itemize}
            \item<1-> $u_iv_i \geqslant 0$
            \item<2-> and $|u_i| \leqslant |v_i|.$
        \end{itemize}
    \end{defi}
    \pause
    \begin{defi}[Graver basis]
        For a given matrix $A,$ the Graver basis $\graver(A)$ is the set of conformally minimal vectors in $(\ker A \cap \Z^n)\backslash \{\vect{0}\}.$
    \end{defi}
\end{frame}

\begin{frame}{Graver complexity}
    \begin{defi*}[Graver basis]
        For a given matrix $A,$ the Graver basis $\graver(A)$ is the set of conformally minimal vectors in $(\ker A \cap \Z^n)\backslash \{\vect{0}\}.$
    \end{defi*}
    \begin{exampleblock}{Graver basis example}
        Let $A = (1 \ 2 \ 1).$ \\
        \pause
        Then $\ker A \cap \Z^3 = \{(x_1, x_2, -x_1-2x_2): x_1, x_2 \in \Z\}.$ \\
        \pause
        Hence, $\graver(A) = \pm \{(0,1,-2), (1,0,-1), (-2,1,0), (1,-1,1)\}.$
    \end{exampleblock}
\end{frame}

\begin{frame}{Graver complexity}
    \begin{defi*}[Graver basis]
        For a given matrix $A,$ the Graver basis $\graver(A)$ is the set of conformally minimal vectors in $(\ker A \cap \Z^n)\backslash \{\vect{0}\}.$
    \end{defi*}
    \begin{fakt}
        The Graver basis $\graver(A)$ is finite. Indeed, note that $\cord$ in a partial order on $\Z^n$ and by Dickson's lemma it's a well-quasi-ordering on $\Z^n.$ Thus, there is no infinite antichains with respect to $(\Z^n, \cord).$
    \end{fakt}
\end{frame}

\begin{frame}{Graver complexity}
    \begin{minipage}{0.4\textwidth}
        \begin{defi}[Brick]
            For any $n$ we write each vector $\vect x \in \Z^{nt}$ as a tuple $\vect x = (\vect x^1, \ldots, \vect x^n)$ of $n$ bricks $\vect x^i \in \Z^t$
        \end{defi}
    \end{minipage}
    \begin{minipage}{0.55\textwidth}
    \begin{center}
        $$
            \vect x = (1, 2, 3, 4, 5, 6)
        $$
        $$
            \big\downarrow n = 3
        $$
        $$
            \left(\vect x^1, \vect x^2, \vect x^3\right) = \left(
                \left[\begin{matrix}
                    1 \\ 2
                \end{matrix}\right],
                \left[\begin{matrix}
                    3 \\ 4
                \end{matrix}\right],
                \left[\begin{matrix}
                    5 \\ 6
                \end{matrix}\right]
            \right)
        $$
    \end{center}
    \end{minipage}
    \begin{defi}[Graver complexity for bimatrix]
        The Graver complexity of an integer bimatrix $A$ is defined to be the largest number $g(A)$ of nonzero bricks $\vect g^i$ in any element $\vect g \in \graver\left(A^{(n)}\right)$ for any $n.$
        The number $g(A)$ is bounded by a constant independent of $n.$ (WIP: proof)
    \end{defi}
\end{frame}

\begin{frame}{Graver complexity}
\end{frame}

% \AtBeginSection[ ]
% {
% \begin{frame}{Outline}
    % \/tableofcontents[currentsection]
% \end{fra/me}
% }

%\AtBeginSubsection[ ]
%{
%\begin{frame}{Outline}
%    \tableofcontents[currentsection]
%\end{frame}
%}

% \AtBeginSubsection[]{
%   \frame<beamer>{ 
    % \frametitle{Outline}   
% 
% \tableofcontents[currentsection,currentsubsection,sectionstyle=show,subsectionstyle=show/shaded,]}
% }



\end{document} 